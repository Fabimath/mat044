\begin{center}
    \huge{
    \textbf{
        Ayudantía 1
    }
    }\\
    \normalsize{
        Fabián Ramírez Díaz
    }
\end{center}
\section*{Problema 1}
Sea $X$ una v.a. con función de densidad $$f(x;\theta) = \pvi{
\theta x^{\theta-1} \text{ si } 0<x<1.\\
0 \text{ en otro caso}
}$$ en donde $\theta >0$. Obtenga el estimador de momentos de $\theta$.
\section*{Problema 2}
Sea $X_1, . . . , X_n$ una muestra aleatoria de la distribución
$\texttt{Ber}(\theta)$, con $\theta$ desconocido. Sea $\gorro{\theta_1} = X_1$ y $\gorro{\theta_2} = \barra{X}$. ¿Qué estimador es mejor?
\section*{Problema 3}
Sea $X\sim\texttt{Poi}(\theta)$. Encuentre el estimador de momentos de $\theta$ y demuestre que es insesgado.
\section*{Problema 4}
Demuestre que el estimador $\est_n := \dfrac{1}{\barra{X}}$ es consistente para el parámetro $\theta$ en la distribución $\texttt{exp}(\theta)$

\vspace{1cm}
\caja{
\begin{center}
\emph{
Puede que haya perdido todo, pero jamás dejaré de pelear por lo que creo.
}\\
\textbf{Son Goku\\ Dragon Ball}
\end{center}
}