\begin{center}
    \huge{
    \textbf{
        Ayudantía 1
    }
    }\\
    \normalsize{
        Fabián Ramírez Díaz
    }
\end{center}

\section*{Contexto}
Un jugador competitivo de Pokemón desea seleccionar los mejores Pokemones para su equipo, para ello dispone de una base de datos que contiene todas las especies, esta base la puede descargar del siguiente \href{https://fabimath.github.io/mat044/ay1/pokemon.csv}{\textcolor{blue}{link}}. Ahora bien él no sabe ni calcular un promedio, por tanto le pide ayuda a un estudiante de ingeniería que lo ayude a resolver su problema. 
\section*{Problema 1}
El jugador nos dice que una de las medidas mas importantes de los Pokemones es su HP, por tanto le gustaría que realizaras un análisis descriptivo de dicha variable indicando cual es la media, mediana y moda de dicha variable. Finalmente quiere que le entregues una sub-base de datos que contenga sólo los Pokemones con HP mayor o igual que el indicador de tendencia central mas grande.
\section*{Problema 2}
Una ves obtenida la sub-base, el jugador nos dice que es tan bueno que puede ganar sin legendarios, por tanto te pide que filtres nuevamente la base dejando en ella solo los no legendarios que cumplan la característica pedida en el Problema 1. Ahora el jugador cree que los otros jugadores escogerán como mínimo 3 Pokemones de tipo fuego ya sea como tipo principal o secundario por tanto el escogerá 3 Pokemones de tipo agua que se encuentren dentro del $75\%$ de Pokemones con mayor cantidad de puntos de Ataque. Entréguele         esa base de datos al profesional.
\section*{Problema 3}
El profesional afirma que dentro de las opciones que le entregamos se tiene que la variable \texttt{Sp. Attack} y \texttt{Sp. 	Defense} son muy similares por no decir correlacionadas. Utilice un estadístico apropiado para sustentar la afirmación del profesional o bien negarla. Interprete.

% Table generated by Excel2LaTeX from sheet 'Hoja1'
\begin{table}[htbp]
	\centering
	\caption{Add caption}
	\begin{tabular}{cccccccc}
		\rowcolor[rgb]{ 1,  .851,  .4}       & \multicolumn{1}{p{7.645em}}{\textbf{Lunes}} & \multicolumn{1}{p{7.285em}}{\textbf{Martes}} & \multicolumn{1}{p{5.355em}}{\textbf{Miercoles}} & \multicolumn{1}{p{5.355em}}{\textbf{Jueves}} & \multicolumn{1}{p{5.355em}}{\textbf{Viernes}} & \multicolumn{1}{p{5.355em}}{\textbf{Sábado}} & \multicolumn{1}{p{5.355em}}{\textbf{Domingo}} \\
		\rowcolor[rgb]{ 1,  .851,  .4} \textbf{1} & \multicolumn{1}{p{7.645em}}{\cellcolor[rgb]{ .988,  .898,  .804}\textbf{Inferencia}} & \cellcolor[rgb]{ .718,  .718,  .718} & \multicolumn{1}{p{5.355em}}{\cellcolor[rgb]{ .988,  .898,  .804}\textbf{Inferencia}} & \cellcolor[rgb]{ .718,  .718,  .718} & \cellcolor[rgb]{ .718,  .718,  .718} & \cellcolor[rgb]{ .718,  .718,  .718} & \cellcolor[rgb]{ .718,  .718,  .718} \\
		\rowcolor[rgb]{ 1,  .851,  .4} \textbf{2} & \multicolumn{1}{p{7.645em}}{\cellcolor[rgb]{ .988,  .898,  .804}\textbf{8:00-9:30}} & \cellcolor[rgb]{ .718,  .718,  .718} & \multicolumn{1}{p{5.355em}}{\cellcolor[rgb]{ .988,  .898,  .804}\textbf{8:00-9:30}} & \cellcolor[rgb]{ .718,  .718,  .718} & \cellcolor[rgb]{ .718,  .718,  .718} & \cellcolor[rgb]{ .718,  .718,  .718} & \cellcolor[rgb]{ .718,  .718,  .718} \\
		\rowcolor[rgb]{ 1,  .851,  .4} \textbf{3} & \cellcolor[rgb]{ .718,  .718,  .718} & \cellcolor[rgb]{ .718,  .718,  .718} & \multicolumn{1}{p{5.355em}}{\cellcolor[rgb]{ 1,  .6,  0}\textbf{Fis140}} & \cellcolor[rgb]{ .718,  .718,  .718} & \multicolumn{1}{p{5.355em}}{\cellcolor[rgb]{ 1,  .6,  0}\textbf{Fis140}} & \cellcolor[rgb]{ .718,  .718,  .718} & \cellcolor[rgb]{ .718,  .718,  .718} \\
		\rowcolor[rgb]{ 1,  .851,  .4} \textbf{4} & \cellcolor[rgb]{ .718,  .718,  .718} & \cellcolor[rgb]{ .718,  .718,  .718} & \multicolumn{1}{p{5.355em}}{\cellcolor[rgb]{ 1,  .6,  0}\textbf{9:45-11:15}} & \cellcolor[rgb]{ .718,  .718,  .718} & \multicolumn{1}{p{5.355em}}{\cellcolor[rgb]{ 1,  .6,  0}\textbf{9:45-11:15}} & \cellcolor[rgb]{ .718,  .718,  .718} & \cellcolor[rgb]{ .718,  .718,  .718} \\
		\rowcolor[rgb]{ 1,  .851,  .4} \textbf{5} & \cellcolor[rgb]{ .718,  .718,  .718} & \cellcolor[rgb]{ .718,  .718,  .718} & \cellcolor[rgb]{ .718,  .718,  .718} & \cellcolor[rgb]{ .718,  .718,  .718} & \cellcolor[rgb]{ .718,  .718,  .718} & \cellcolor[rgb]{ .718,  .718,  .718} & \cellcolor[rgb]{ .718,  .718,  .718} \\
		\rowcolor[rgb]{ 1,  .851,  .4} \textbf{6} & \cellcolor[rgb]{ .718,  .718,  .718} & \cellcolor[rgb]{ .718,  .718,  .718} & \cellcolor[rgb]{ .718,  .718,  .718} & \cellcolor[rgb]{ .718,  .718,  .718} & \cellcolor[rgb]{ .718,  .718,  .718} & \cellcolor[rgb]{ .718,  .718,  .718} & \cellcolor[rgb]{ .718,  .718,  .718} \\
		\rowcolor[rgb]{ 1,  .851,  .4} \textbf{7} & \multicolumn{1}{p{7.645em}}{\cellcolor[rgb]{ 1,  .6,  0}\textbf{Opti}} & \cellcolor[rgb]{ .718,  .718,  .718} & \multicolumn{1}{p{5.355em}}{\cellcolor[rgb]{ 1,  .6,  0}\textbf{Opti}} & \cellcolor[rgb]{ .714,  .843,  .659}\textbf{3:00} & \multicolumn{1}{p{5.355em}}{\cellcolor[rgb]{ 1,  .6,  0}\textbf{Micro}} & \cellcolor[rgb]{ .718,  .718,  .718} & \cellcolor[rgb]{ .718,  .718,  .718} \\
		\rowcolor[rgb]{ 1,  .851,  .4} \textbf{8} & \multicolumn{1}{p{7.645em}}{\cellcolor[rgb]{ 1,  .6,  0}\textbf{2:00-3:30}} & \cellcolor[rgb]{ .718,  .718,  .718} & \multicolumn{1}{p{5.355em}}{\cellcolor[rgb]{ 1,  .6,  0}\textbf{2:00-3:30}} & \multicolumn{1}{p{5.355em}}{\cellcolor[rgb]{ .714,  .843,  .659}\textbf{Practica}} & \multicolumn{1}{p{5.355em}}{\cellcolor[rgb]{ 1,  .6,  0}\textbf{2:00-17:10}} & \cellcolor[rgb]{ .718,  .718,  .718} & \cellcolor[rgb]{ .718,  .718,  .718} \\
		\rowcolor[rgb]{ 1,  .851,  .4} \textbf{9} & \cellcolor[rgb]{ .718,  .718,  .718} & \cellcolor[rgb]{ .718,  .718,  .718} & \cellcolor[rgb]{ .718,  .718,  .718} & \cellcolor[rgb]{ .718,  .718,  .718} & \cellcolor[rgb]{ 1,  .6,  0} & \cellcolor[rgb]{ .718,  .718,  .718} & \cellcolor[rgb]{ .718,  .718,  .718} \\
		\rowcolor[rgb]{ 1,  .851,  .4} \textbf{10} & \cellcolor[rgb]{ .718,  .718,  .718} & \cellcolor[rgb]{ .718,  .718,  .718} & \cellcolor[rgb]{ .718,  .718,  .718} & \cellcolor[rgb]{ .718,  .718,  .718} & \cellcolor[rgb]{ 1,  .6,  0} & \cellcolor[rgb]{ .718,  .718,  .718} & \cellcolor[rgb]{ .718,  .718,  .718} \\
		\rowcolor[rgb]{ 1,  .851,  .4} \textbf{11} & \cellcolor[rgb]{ .718,  .718,  .718} & \cellcolor[rgb]{ .718,  .718,  .718} & \cellcolor[rgb]{ .718,  .718,  .718} & \cellcolor[rgb]{ .718,  .718,  .718} & \cellcolor[rgb]{ .718,  .718,  .718} & \cellcolor[rgb]{ .718,  .718,  .718} & \cellcolor[rgb]{ .718,  .718,  .718} \\
		\rowcolor[rgb]{ 1,  .851,  .4} \textbf{12} & \multicolumn{1}{p{7.645em}}{\cellcolor[rgb]{ .918,  .6,  .6}\textbf{18:40 - 19:50}} & \multicolumn{1}{p{7.285em}}{\cellcolor[rgb]{ .918,  .6,  .6}\textbf{18:30 - 19:40}} & \cellcolor[rgb]{ .718,  .718,  .718} & \cellcolor[rgb]{ .718,  .718,  .718} & \cellcolor[rgb]{ .718,  .718,  .718} & \cellcolor[rgb]{ .718,  .718,  .718} & \cellcolor[rgb]{ .718,  .718,  .718} \\
		\rowcolor[rgb]{ 1,  .851,  .4} \textbf{13} & \multicolumn{1}{p{7.645em}}{\cellcolor[rgb]{ .918,  .6,  .6}\textbf{Preusm}} & \multicolumn{1}{p{7.285em}}{\cellcolor[rgb]{ .918,  .6,  .6}\textbf{Lec}} & \multicolumn{1}{p{5.355em}}{\cellcolor[rgb]{ .918,  .6,  .6}\textbf{Mat044}} & \multicolumn{1}{p{5.355em}}{\cellcolor[rgb]{ .918,  .6,  .6}\textbf{Mat044}} & \cellcolor[rgb]{ .718,  .718,  .718} & \cellcolor[rgb]{ .718,  .718,  .718} & \cellcolor[rgb]{ .718,  .718,  .718} \\
		\rowcolor[rgb]{ 1,  .851,  .4} \textbf{14} & \cellcolor[rgb]{ .718,  .718,  .718} & \cellcolor[rgb]{ .718,  .718,  .718} & \cellcolor[rgb]{ .918,  .6,  .6}\textbf{19:00} & \cellcolor[rgb]{ .918,  .6,  .6}\textbf{19:00} & \cellcolor[rgb]{ .718,  .718,  .718} & \cellcolor[rgb]{ .718,  .718,  .718} & \cellcolor[rgb]{ .718,  .718,  .718} \\
	\end{tabular}%
	\label{tab:addlabel}%
\end{table}%



% \vspace{1cm}\caja{\begin{center} Puede que haya perdido todo, pero jamás dejaré de pelear por lo que creo. \\ \textbf{Son Goku\\ Dragon Ball} \end{center} }